\documentclass[12pt,prb,aps]{revtex4-1}
\usepackage {amsmath}
\usepackage{amssymb}
\pdfoutput = 1 
\usepackage {graphicx}
\newcommand{\bomega}{\mbox{\boldmath$\omega$}}
\allowdisplaybreaks

\begin{document}

\title{Time-Dependent Parallel Electron Particle and Energy Transport in  a Magnetized Plasma of Arbitrary Collisonality}
\author{T.~Wang and R.~Fitzpatrick\,\footnote{rfitzp@utexas.edu}}
\affiliation{Institute for Fusion Studies,  Department of Physics,  University of Texas at Austin,  Austin TX 78712, USA}
\maketitle

\section{Introduction}
In Ref.~\onlinecite{haz},  the steady-state transport of electron number density and energy parallel to the magnetic field
of a magnetized, weakly-coupled, electron-ion plasma of arbitrary collisionality was investigated  in slab geometry   by solving a simplified one-dimensional kinetic equation for the electron distribution function that employed a Bhatnagar-Gross-Krook (BGK)  electron-electron collision
operator.\cite{krook}  The resulting model was able to successfully reproduce standard results for the the electron heat flux in both the short mean-free-path and
the long mean-free-path limits. In the short mean-free-path limit, electron energy transport was found to be local and diffusive in nature, whereas the transport was found to be non-local and convective in the
long mean-free-path limit. Somewhat surprisingly, electron particle transport was found to be local and diffusive at all collisionalities. 
The aim of this paper is to generalize the analysis of Ref.~\onlinecite{haz} by  incorporating a model electron-ion collision operator,
 including a self-consistent calculation of the parallel electric field, and taking time dependence into account.

\section{Basic Model}

\subsection{Electron Distribution Function}
Let $f_e(t,{\bf x},{\bf v})$ be the ensemble-averaged electron distribution function. Here, $t$ denotes time, ${\bf x}=(x_1,\, x_2,\, x_3)$
is a position vector,  $x_1$, $x_2$, $x_3$ are Cartesian coordinates that are defined such that the 
$x_3$-axis is parallel to the local equilibrium magnetic field, and ${\bf v}$ is the electron velocity. 
Let us write
\begin{equation}\label{e22}
f_e(t,{\bf x},{\bf v})= n_e\,F(v_1)\,F(v_2)\left[F(v_3)+ f (t,x_3,v_3)\right],
\end{equation}
where 
\begin{equation}
F(v) = \frac{\exp(-v^2/v_{t\,e}^{\,2})}{\pi^{1/2}\,v_{t\,e}},
\end{equation}
and 
$|f/F|\ll 1$. Here,  $n_e$ is the unperturbed electron number density, 
\begin{equation}
v_{t\,e} = \sqrt{\frac{2\,T_e}{m_e}}
\end{equation}
the electron thermal velocity, $m_e$  the electron mass,  and $T_e$ the unperturbed electron temperature (measured in energy units). Note that we are assuming that the electron distribution function remains relatively close to a Maxwellian distribution. 

\subsection{Electron-Electron Collision Operator}\label{see}
The electron-electron collision operator is modeled as a BGK operator:\,\cite{haz,krook}
\begin{align}
C_{ee}(f_e) &= -\nu_{ee}\,F(v_1)\,F(v_2)\,\biggr\{n_e\,F(v_3)+n_e\,f(t,x_3,v_3)\phantom{\frac{a}{b}}\nonumber\\[0.5ex]
&\phantom{=}\left.- \frac{[n_e+\delta n_e(t,x_3)]\,m_e^{\,1/2}}{\pi^{1/2}\,(2\,[T_e+\delta T_e(t,x_3)])^{1/2}}\,\exp\left[-\frac{[v_3-V_e(t,x_3)]^{2}\,m_e}{2\,[T_e+\delta T_e(t,x_3)]}\right]\right\}.
\end{align}
Here, $\nu_{ee}$ is the electron-electron collision frequency. Moreover, $|\delta n_e/n_e|\ll 1$, $|V_e/v_3|\ll 1$,  and $|\delta T_e/T_e|\ll 1$. It can be seen that the operator acts to relax the distribution function to the 
perturbed Maxwellian
\begin{equation}
F(v_1)\,F(v_2)\,\frac{[n_e+\delta n_e(t,x_3)]\,m_e^{\,1/2}}{\pi^{1/2}\,(2\,[T_e+\delta T_e(t,x_3)])^{1/2}}\,\exp\left[-\frac{[v_3-V_3(t,x_3)]^{2}\,m_e}{2\,[T_e+\delta T_e(t,x_3)]}\right].
\end{equation}
Note that we are working in an assumed common electron-ion rest frame.
Expanding the collision operator, and only retaining terms that are first order in perturbed quantities, we obtain
\begin{align}\label{e24}
C_{ee}(f_e) &=- \nu_{ee}\,n_e\,F(v_1)\,F(v_2)\,\left\{ f(t,x_3,v_3) - \left[\frac{\delta n_e(t,x_3)}{n_e} +\frac{V_e(t,x_3)}{v_{t\,e}} \,\frac{2\,v_3}{v_{t\,e}}\right.\right.\nonumber\\[0.5ex]
&\phantom{=}\left.\left.+\frac{\delta T_e(t,x_3)}{T_e}\left(\frac{v_3^{\,2}}{v_{t\,e}^{\,2}}-\frac{1}{2}\right)\right]F(v_3)\right\}.
\end{align}

Now, in order for the electron-electron collision operator to conserve the number of electrons, we require that
\begin{equation}
\int\!\!\int\!\!\int C_{ee}(f_e) \,d^3{\bf v} = 0,
\end{equation}
which yields
\begin{equation}\label{dne}
\frac{\delta n_e(t,x_3)}{n_e} =\int_{-\infty}^\infty f(t,x_3,v_3)\,dv_3.
\end{equation}
Here, $\delta n_e$ is the perturbed electron number density. 

Likewise, in order for the electron-electron collision operator to conserve electron momentum, we need
\begin{equation}\label{e27}
\int\!\!\int\!\!\int {\bf v}\,C_{ee}(f_e)\, d^3{\bf v}= {\bf 0}.
\end{equation}
It is easily seen that 
$\int\!\!\int\!\!\int v_1\,C_{ee}(f_e)\,d^3{\bf v} =\int\!\!\int\!\!\int v_2\,C_{ee}(f_e)\,d^3{\bf v} = 0$.
Thus, we require 
\begin{equation}\label{eyy}
\int\!\!\int\!\!\int v_3\,C_{ee}(f_e)\, d^3{\bf v} = 0,
\end{equation}
which yields
\begin{equation}\label{e29}
 V_e(t,x_3) = \int_{-\infty}^\infty v_3\,f(t,x_3,v_3)\,dv_3.
\end{equation}
Here, $V_e$ is the perturbed parallel drift velocity of the electrons with respect to the ions.

Finally, in order for the electron-electron collision operator to conserve electron energy, we require
\begin{equation}\label{e20}
\int\!\!\int\!\!\int v^2\,C_{ee}(f_e)\, d^3{\bf v} = 0.
\end{equation}
It is easily seen that 
$\int\!\!\int\!\!\int v_1^{\,2}\,C_{ee}(f_e) \,d^3{\bf v} = \int\!\!\int\!\!\int v_2^{\,2}\,C_{ee}(f_e)\,d^3{\bf v} =0$.
Thus, we need 
\begin{equation}\label{e36s}
\int\!\!\int\!\!\int v_3^{\,2}\,C_{ee}(f_e) \,d^3{\bf v} = 0,
\end{equation}
which yields
\begin{equation}\label{dte}
\frac{\delta T_e(t,x_3)}{T_e} = 2\,\int_{-\infty}^\infty \left(\frac{v_3^{\,2}}{v_{t\,e}^{\,2}}-\frac{1}{2}\right) f(t,x_3,v_3)\,dv_3.
\end{equation}
Here, $\delta T_e$ is the perturbed electron temperature.
Thus, the electron-electron collision operator, (\ref{e24}), is now fully specified in terms of the perturbed electron distribution function,
$f(t,x_3,v_3)$. 

\subsection{Electron-Ion Collision Operator}
By analogy with the analysis in the previous subsection, our model electron-ion collision operator is written
\begin{align}
C_{ei}(f_e) &=- \nu_{ei}\,n_e\,F(v_1)\,F(v_2)\,\biggr\{ f(t,x_3,v_3) \nonumber\\[0.5ex]&\phantom{=}\left.- \left[\frac{\delta n_e(t,x_3)}{n_e} +\frac{\delta T_e(t,x_3)}{T_e}\,\left(\frac{v_3^{\,2}}{v_{t\,e}^{\,2}}-\frac{1}{2}\right)\right]F(v_3)\right\},
\end{align}
where $\nu_{ei}$ is the electron-ion collision frequency. Note that this collision operator conserves the number of electrons, as well as the electron energy (because the ions are treated as
infinitely massive with respect to the electrons), but does not conserve electron momentum (as a consequence of momentum transferred to the ions via collisions). Note, finally, that the
ion fluid is stationary in the infinite mass limit. 

\subsection{Electron Kinetic Equation}
The  ensemble-averaged electron kinetic equation that governs the transport of electron number density
and energy parallel to the magnetic field can be written\,\cite{haz,rf0}
\begin{equation}
\frac{\partial f_e}{\partial t}+v_3\,\frac{\partial f_e}{\partial x_3} -\frac{e}{m_e}\,E_3\,\frac{\partial f_e}{\partial v_3} = C_{ei}(f_e) + C_{ee}(f_e) + S({\bf x},{\bf v}).
\end{equation}
Here, we are assuming that the plasma is subject to a perturbed parallel electric
field, $E_3(t,x_3)$. Moreover, the source term in the kinetic equation takes the form 
\begin{equation}
S(t,{\bf x},{\bf v}) = n_e\,F(v_1)\,F(v_2)\,F(v_3)\left[S_0(t,x_3) +S_2(t,x_3)\left(\frac{v_3^{\,2}}{v_{t\,e}^{\,2}}-\frac{1}{2}\right) \right],
\end{equation}
where $S_0(t,x_3)$ represents a particle source, and $S_2(t,x_3)$ represents an energy source. 

Linearizing the kinetic equation, and integrating over $v_1$ and $v_2$, we obtain
\begin{equation}\label{e40}
\frac{\partial f}{\partial t}+v_3\,\frac{\partial f}{\partial x_3} - \langle C_{ei}( f)\rangle- \langle C_{ee}(f)\rangle = \left[
S_0 +S_1\,\frac{2\,v_3}{v_{t\,e}} +S_2 \left(\frac{v_3^{\,2}}{v_{t\,e}^{\,2}}-\frac{1}{2}\right)\right]F(v_3),
\end{equation}
where
\begin{align}\label{cee}
\langle C_{ee}(f)\rangle&= \frac{1}{n_e}\int_{-\infty}^\infty\!\int_{-\infty}^\infty C_{ee}(f_e)\,dv_1\,dv_2\nonumber\\[0.5ex]&= -\nu_{ee}\left\{f-\left[\frac{\delta n_e}{n_e}+\frac{ V_e}{v_{t\,e}}\,\frac{2\,v_3}{v_{t\,e}}
+\frac{\delta T_e}{T_e}\left(\frac{v_3^{\,2}}{v_{t\,e}^{\,2}}-\frac{1}{2}\right)\right]F(v_3)\right\},\\[0.5ex]
\langle C_{ei}(f)\rangle&= \frac{1}{n_e}\int_{-\infty}^\infty\!\int_{-\infty}^\infty C_{ei}(f_e)\,dv_1\,dv_2\nonumber\\[0.5ex]&=-\nu_{ei}\left\{f-\left[\frac{\delta n_e}{n_e}
+\frac{\delta T_e}{T_e}\left(\frac{v_3^{\,2}}{v_{t\,e}^{\,2}}-\frac{1}{2}\right)\right]F(v_3)\right\},\label{cei}\\[0.5ex]
S_1(t,x_3) &= - \frac{e\, E_3(t,x_3)}{m_e\,v_{t\,e}}.\label{s1}
\end{align}

\subsection{Poisson-Maxwell Equation}
Assuming that the ions constitute a uniform neutralizing background, the perturbed parallel electric field is related to the perturbed electron number
density according to
\begin{equation}\label{pois}
\frac{\partial  E_3}{\partial x_3}=-\frac{e\,\delta n_e(t,x_3)}{\epsilon_0}.
\end{equation}

\subsection{Heat Flux}
The flux of parallel electron kinetic energy is defined\,\cite{rf0}
\begin{equation}
{\bf q}_\parallel(t,{\bf x}) =  \int\!\int\!\int \frac{1}{2}\,m_e\,(v_3-V_e)^2\,({\bf v}-{\bf V})\,f_e(t,{\bf x},{\bf v})\,d^3{\bf v}.
\end{equation}
It is easily demonstrated that, to first order in small quantities, $q_{\parallel\,1}=q_{\parallel\,2}=0$, and
\begin{equation}\label{q3}
q_{\parallel\,3}(t,x_3) = \frac{1}{2}\,m_e\,n_e\int_{-\infty}^\infty v_3\left(v_3^{\,2}
-\frac{3}{2}\,v_{t\,e}^{\,2}\right)f(t,x_3,v_3)\,dv_3.
\end{equation}

\section{Fourier-Laplace Transform Solution of Electron Kinetic Equation}
\subsection{Normalization}
Let
\begin{align}
\nu_e= \nu_{ee}+\nu_{ei}
\end{align}
be the total electron collision frequency, and
let
\begin{equation}
l_e= \frac{v_{t\,e}}{\nu_e}
\end{equation}
be the mean-free-path between collisions. 
Let us adopt the following normalizations: 
$\hat{t} = \nu_{e}\,t$,
$\hat{x}= x_3/l_e$,
$u = v_3/v_{t\,e}$,
$\hat{f}= v_{t\,e}\,f$,
$\delta\hat{n}_e= \delta n_e/n_e$,
$\hat{V}_e= V_e/v_{t\,e}$,
$\delta\hat{T}_e= \delta T_e/T_e$,
$\hat{S}_0 = S_0/\nu_{e}$,
$\hat{S}_1 = S_1/\nu_{e}$,
$\hat{S}_2= S_2/\nu_{e}$, and
$\hat{q}_e=q_{\parallel\,3}/(n_e\,T_e\,v_{t\,e})$.
 
The electron kinetic equation, (\ref{e40}), takes the normalized form
\begin{align}\label{e37a}
\frac{\partial\hat{f}}{\partial \hat{t}}+u\,\frac{\partial\hat{f}}{\partial\hat{x}} + \hat{f}
= \left[(\delta\hat{n}_e+\hat{S}_0)+(\mu_e\,\hat{V}_e+\hat{S}_1)\,2\,u+(\delta\hat{T}_e+\hat{S}_2)\left(u^2-\frac{1}{2}\right)\right] F_M,
\end{align}
where
\begin{align}
F_M(u) &=\frac{\exp(-u^2)}{\pi^{1/2}},\\[0.5ex]
\mu_e&= \frac{\nu_{ee}}{\nu_{ee}+\nu_{ei}}.
\end{align}
Here, use has been made of Eqs.~(\ref{cee}) and (\ref{cei}). Furthermore, Eqs.~(\ref{dne}), (\ref{e29}), (\ref{dte}), and (\ref{q3}) yield
\begin{align}\label{e28}
\delta\hat{n}_e(\hat{t},\hat{x})&=\int_{-\infty}^\infty \hat{f}(\hat{t},\hat{x},u)\,du,\\[0.5ex]
\hat{V}_e(\hat{t},\hat{x})&= \int_{-\infty}^\infty u\,\hat{f}(\hat{t},\hat{x},u)\,du,\\[0.5ex]
\delta\hat{T}_e(\hat{t},\hat{x})&= 2\int_{-\infty}^\infty \left(u^2-\frac{1}{2}\right)f(\hat{t},\hat{x},u)\,du,\label{e30}\\[0.5ex]
\hat{q}_e(\hat{t},\hat{x})&= \int_{-\infty}^\infty u\left(u^2-\frac{3}{2}\right)\hat{f}(\hat{t},\hat{x},u)\,du.
\end{align}
Finally, Eqs.~(\ref{s1}) and (\ref{pois}) give
\begin{equation}\label{e44a}
2\,\hat{\lambda}_D^{\,2}\,\frac{\partial\hat{S}_1}{\partial \hat{x}} = \delta\hat{n}_e,
\end{equation}
where
\begin{equation}
\hat{\lambda}_D = \frac{\lambda_D}{l_e}
\end{equation}
and
\begin{equation}
\lambda_D = \left(\frac{\epsilon_0\,T_e}{n_e\,e^2}\right)^{1/2}
\end{equation}
is the Deybe length.\cite{rf0} Note that $\hat{\lambda}_D$ is necessarily a small parameter in a weakly-coupled plasma.\cite{rf0}

\subsection{Fluid Equations}
Taking $\int_{-\infty}^\infty (\ref{e37a})\,du$, we obtain the electron continuity equation,
\begin{equation}\label{econt}
\frac{\partial\delta\hat{n}_e}{\partial\hat{t}}+\frac{\partial\hat{V}_e}{\partial\hat{x}} = \hat{S}_0.
\end{equation}
Taking $\int_{-\infty}^\infty u\,(\ref{e37a})\,du$, we obtain the electron momentum conservation equation, 
\begin{equation}\label{eforce}
\frac{\partial\hat{V}_e}{\partial\hat{t}}+\frac{1}{2}\,\frac{\partial}{\partial\hat{x}}(\delta\hat{n}_e+\delta\hat{T}_e) + (1-\mu_e)\,\hat{V}_e = \hat{S}_1.
\end{equation}
Finally, taking $2\,\int_{-\infty}^\infty (u^2-1/2)\,(\ref{e37a})\,du$, we obtain
 the electron  energy conservation equation,
\begin{equation}\label{eenergy}
\frac{\partial\delta T_e}{\partial\hat{t}} +2\,\frac{\partial}{\partial\hat{x}}(\hat{V}_e+\hat{q}_e)= \hat{S}_2.
\end{equation}


\subsection{Fourier-Laplace Transformation}
Let
\begin{align}
\bar{f}(g,\hat{k},u) &= \frac{1}{\sqrt{2\pi}}\int_{-\infty}^\infty\left(\int_0^\infty \hat{f}(\hat{t},\hat{x},u)\,{\rm e}^{-g\,\hat{t}}\,d\hat{t}\right){\rm e}^{-{\rm i}\,\hat{k}\,\hat{x}}\,d\hat{x},\\[0.5ex]
\delta\bar{n}_e(g,\hat{k}) &=\frac{1}{\sqrt{2\pi}} \int_{-\infty}^\infty\left(\int_0^\infty \delta \hat{n}_e(\hat{t},\hat{x})\,{\rm e}^{-g\,\hat{t}}\,d\hat{t}\right){\rm e}^{-{\rm i}\,\hat{k}\,\hat{x}}\,d\hat{x},
\label{e41x}
\end{align}
et cetera.
Here,
$\hat{k}= k\,l_e$,
where $k$ is the unormalized wavenumber.
If we operate on Eqs.~(\ref{e37a}) and (\ref{e44a}) with $\int_{-\infty}^\infty\int_0^\infty[(\cdots)\,{\rm e}^{-g\,\hat{t}}\,d\hat{t}]\,{\rm e}^{-{\rm i}\,\hat{k}\,\hat{x}}\,d\hat{x}$, 
and combine the
resulting equations, then we obtain
\begin{align}\label{e55x}
(g+{\rm i}\,\hat{k}\,u+1)\,\bar{f}&= \left[(\delta\bar{n}_e+\bar{S}_0)+\left(\mu_e\,\bar{V}_e+\frac{\delta\bar{n}_e}{2\,{\rm i}\,\hat{k}\,\hat{\lambda}_D^{\,2}}\right)2\,u+(\delta\bar{T}_e+\bar{S}_2)\left(u^2-\frac{1}{2}\right)\right] F_M.
\end{align}
Here, we are assuming that all perturbed quantities are zero for $\hat{t}<0$. 

\subsection{Fourier-Laplace Transformed Fluid Equations}
Taking $\int_{-\infty}^\infty (\ref{e55x})\,du$, we obtain the Fourier-Laplace transformed electron continuity equation,
\begin{equation}\label{e34}
g\,\delta\bar{n}_e+{\rm i}\,\hat{k}\,\bar{V}_e =\bar{S}_0.
\end{equation}
Taking $\int_{-\infty}^\infty u\,(\ref{e55x})\,du$, we obtain the Fourier-Laplace transformed electron momentum conservation equation,
\begin{equation}\label{e35}
(g+1-\mu_e)\,\bar{V}_e +\frac{{\rm i}\,\hat{k}}{2}\left({\mit\Lambda}_D\,\delta\bar{n}_e+\delta\bar{T}_e\right)=0,
\end{equation}
where
\begin{equation}
{\mit\Lambda}_D (\hat{k})= \frac{1+(\hat{k}\,\hat{\lambda}_D)^2}{(\hat{k}\,\hat{\lambda}_D)^2}= \frac{1+(k\,\lambda_D)^2}{(k\,\lambda_D)^2}.
\end{equation}
Finally, taking $2\int_{-\infty}^\infty(u^2-1/2)\,(\ref{e55x})\,du$, we obtain the Fourier-Laplace transformed electron energy conservation equation,
\begin{equation}\label{e36}
g\,\delta\bar{T}_e+ 2\,{\rm i}\,\hat{k}\,(\bar{V}_e+\bar{q}_e) =\bar{S}_2.
\end{equation}

\subsection{Reformulation}
Equation~(\ref{e55x}) can be rearranged to give
\begin{align}\label{e37}
\bar{f}(g,\hat{k},u)&=(1+g)^{-1}\left\{\left(\delta\bar{n}_e+\bar{S}_0\right)+\left[\mu_e\,\bar{V}_e+\frac{(1+g)\,({\mit\Lambda}_D-1)\,\delta\bar{n}_e}{2\,\xi}\right]2\,u
\right.\nonumber\\[0.5ex]
&\phantom{=}\left.+\left(\delta\bar{T}_e+\bar{S}_2\right)\left(u^2-\frac{1}{2}\right)\right\}\left(\frac{-\xi}{u-\xi}\right)F_M,
\end{align}
where
\begin{equation}
\xi (g,\hat{k})= \frac{{\rm i}\,(1+g)}{\hat{k}}= \frac{{\rm i}\,(1+g)}{k\,l_e}.
\end{equation}
Likewise, the fluid equations, (\ref{e34}), (\ref{e35}), and (\ref{e36}), can be re-expressed in the forms
\begin{align}\label{e39}
\delta\bar{n}_e+\bar{S}_0&= (1+g)\left(\delta\bar{n}_e-\xi^{-1}\,\bar{V}_e\right),\\[0.5ex]
\mu_e\,\bar{V}_e+\frac{(1+g)\,({\mit\Lambda}_D-1)\,\delta\bar{n}_e}{2\,\xi}&= (1+g)\left[\bar{V}_e-\frac{\xi^{-1}}{2}\,(\delta\bar{n}_e+\delta\bar{T}_e)\right],\\[0.5ex]
\delta\bar{T}_e+\bar{S}_2&= (1+g)\left[\delta \bar{T}_e-2\,\xi^{-1}\,(\bar{V}_e+\bar{q}_e)\right].\label{e41}
\end{align}
It follows that
\begin{align}\label{e42}
\xi\left[\mu_e\,\bar{V}_e+\frac{(1+g)\,({\mit\Lambda}_D-1)\,\delta\bar{n}_e}{2\,\xi}\right]& = g\,\xi^2\,\delta\bar{n}_e-\frac{1}{2}\,(1+g)\,(\delta \bar{n}_e+\delta\bar{T}_e)
-\xi^2\,\hat{S}_0,\\[0.5ex]
\bar{q}_e&= (1+g)^{-1}\,\xi\left[-g\left(\delta\bar{n}_e-\frac{\delta\bar{T}_e}{2}\right)+\left(\bar{S}_0-\frac{\bar{S}_2}{2}\right)\right].\label{e43}
\end{align}

\subsection{Modified Plasma Dispersion Function}
Let
\begin{equation}\label{e44}
Z_n(\xi)= \int_{-\infty}^\infty u^n\left(\frac{-\xi}{u-\xi}\right)F_M(u)\,du.
\end{equation}
It is easily demonstrated that 
\begin{equation}
Z_{n+1}= \xi\,(Z_n-I_{n}),
\end{equation}
where
\begin{equation}
I_n = \frac{1}{\pi^{1/2}}\int_{-\infty}^\infty u^n\,\exp(-u^2)\,du.
\end{equation}
Now, $I_0=1$, and $I_1=0$, so
\begin{align}\label{ez1}
Z_1 &= \xi\,(Z_0-I_0) = \xi\,Z_0-\xi,\\[0.5ex]
Z_2&= \xi\,(Z_1-I_1) =  \xi^2\,Z_0-\xi^2.\label{ez2}
\end{align}
Note that 
\begin{equation}
Z_0(\xi)= -\xi\,\bar{Z}(\xi),
\end{equation}
where
\begin{equation}
\bar{Z}(\xi) = \frac{1}{\pi^{1/2}}\int_{-\infty}^\infty \frac{{\rm e}^{-u^2}}{u-\xi}\,du
\end{equation}
is related to the plasma dispersion function.\cite{rf0,fc}
In fact, it can be shown that\,\cite{rf0}
\begin{equation}
\bar{Z}(\xi)= {\rm i}\,\pi^{1/2}\,w(\xi)
\end{equation}
for ${\rm Im}(\xi)>0$, and 
\begin{equation}
\bar{Z}(\xi)={\rm i}\,\pi^{1/2}\,w(\xi)-2\,{\rm i}\,\pi^{1/2}\,\exp(-\xi^2) 
\end{equation}
for ${\rm Im}(\xi)<0$.
Here,
\begin{equation}
w(\xi)= \exp(-\xi^2)\,{\rm erfc}(-{\rm i}\,\xi)
\end{equation}
is a so-called Faddeeva function (alternatively known as a Kramp function),\cite{as} and 
${\rm erfc}(z)$ is the complementary error function.\cite{as} 

Now,\cite{rf0,as}
\begin{equation}
w(\xi)=1+\frac{2\,{\rm i}\,\xi}{\pi^{1/2}}+{\cal O}(\xi^2)
\end{equation}
in the limit $|\xi|\ll 1$, whereas
\begin{equation}
w(\xi) = \sigma\,\exp(-\xi^2)+ \frac{{\rm i}}{\pi^{1/2}}\left[\frac{1}{\xi} + \frac{1}{2\,\xi^3}+\frac{3}{4\,\xi^5}+\frac{15}{8\,\xi^7}+
\frac{105}{16\,\xi^{9}}+{\cal O}\left(\frac{1}{\xi^{11}}\right)\right]
\end{equation}
in the limit $|\xi|\rightarrow\infty$. 
Here,
\begin{equation}
\sigma = \left\{\begin{array}{lll} 0 &~~~~&\mbox{$\xi_i>|\xi_r|^{-1}$}\\[0.25ex]
1 &&\mbox{$|\xi_i| < |\xi_r|^{-1}$}\\[0.25ex]
2&& \mbox{$\xi_i< -|\xi_r|^{-1}$}
\end{array}
\right.,
\end{equation}
where $\xi=\xi_r+{\rm i}\,\xi_i$, and $\xi_r$ and $\xi_i$ are both real. 
It follows that
\begin{equation}\label{e63x}
Z_0(\xi) = -{\rm i}\,\pi^{1/2}\,{\rm sgn}(\xi_i)\,\xi +2\,\xi^2+{\cal O}(\xi^3)
\end{equation}
in the limit $|\xi|\ll 1$, whereas 
\begin{equation}\label{e64x}
Z_0(\xi) =- {\rm i}\,\pi^{1/2}\,\sigma'\,\xi\,\exp(-\xi^2) + 1 + \frac{1}{2\,\xi^2}+\frac{3}{4\,\xi^4}+\frac{15}{8\,\xi^6}+\frac{105}{16\,\xi^8}+{\cal O}\left(\frac{1}{\xi^{10}}\right)
\end{equation}
in the limit $|\xi|\gg 1$,
where
\begin{equation}
\sigma' = \left\{\begin{array}{lll} 0 &~~~~&\mbox{$|\xi_i|>|\xi_r|^{-1}$}\\[0.25ex]
1 &&\mbox{$0<\xi_i < |\xi_r|^{-1}$}\\[0.25ex]
-1&& \mbox{$-|\xi_r|^{-1}<\xi_i<0$}
\end{array}
\right..
\end{equation}

\subsection{Fourier-Laplace Transformed Electron Heat Flux}
Equations~(\ref{e28}), (\ref{e37}), and (\ref{e44}) can be combined to give
\begin{align}\label{e51}
(1+g)\,\delta\bar{n}_e&=\left[(\delta\bar{n}_e+\bar{S}_0)-\frac{1}{2}\,(\delta\bar{T}_e+\bar{S}_2)\right]Z_0+\left[\mu_e\,\bar{V}_e+\frac{(1+g)\,({\mit\Lambda}_D-1)\,\delta\bar{n}_e}{2\,\xi}\right]2\,Z_1\nonumber\\[0.5ex]
&\phantom{=}
+(\delta\bar{T}_e+\bar{S}_2)\,Z_2.
\end{align}
Equations~(\ref{e42}), (\ref{ez1}), (\ref{ez2}), and (\ref{e51})
yield
\begin{equation}\label{e54}
\delta\bar{T}_e = \frac{[2\,\xi^2-(2\,\xi^2-1)\,Z_0]\,[\bar{S}_0-\bar{S}_2/2-g\,\delta\bar{n}_e]}{
(\xi^2-1-g)-(\xi^2-3/2-g)\,Z_0}.
\end{equation}
Finally, Eqs.~(\ref{e43}) and (\ref{e54}) give\,\cite{haz}
\begin{equation}\label{e55}
\bar{q}_e(g,\hat{k}) = \xi\,G(\xi)\,\delta\bar{T}_e(g,\hat{k}),
\end{equation}
where
\begin{equation}
G(\xi) = \frac{(\xi^2-1)-(\xi^2-3/2)\,Z_0}{2\,\xi^2-(2\,\xi^2-1)\,Z_0}.
\end{equation}
Note that the electron heat flux only depends on the perturbed electron temperature, and is independent of both the
perturbed electron number density and the electron drift velocity. 
It follows from Eqs.~(\ref{e63x}) and (\ref{e64x}) that
\begin{equation}
G(\xi) =-{\rm i}\,\frac{{\rm sgn}(\xi_i)}{\pi^{1/2}\,\xi}+{\cal O}(1)
\end{equation}
in the limit $|\xi|\ll 1$, and 
\begin{equation}
G(\xi)= \frac{3}{4\,\xi^2} + \frac{3}{2\,\xi^4} + {\cal O}(\xi^{-6}) 
\end{equation}
in the limit $|\xi|\gg 1$. 

\subsection{Fourier-Laplace Transformed Electron Fluid Quantities}
The Fourier-Laplace transformed fluid equations, (\ref{e39})--(\ref{e41}), combined with Eq.~(\ref{e55}), yield
\begin{align}\label{e57}
g_1\,\delta\bar{n}_e-\xi^{-1}\,\bar{V}_e&= \frac{\bar{S}_0}{1+g},\\[0.5ex]
-{\mit\Lambda}_D\,\delta\bar{n}_e+g_2\,\xi^{-1}\,\bar{V}_e-\delta\bar{T}_e&=0,\\[0.5ex]
-\xi^{-1}\,\bar{V}_e-\left(G-\frac{g_1}{2}\right)\delta\bar{T}_e&= \frac{\bar{S}_2}{2\,(1+g)},\label{e59}
\end{align}
where
\begin{align}
g_1(g)&= \frac{g}{1+g},\\[0.5ex]
g_2(g,\hat{k})&=\frac{2\,(1-\mu_e+g)\,\xi^2}{1+g}.
\end{align}
Finally, Eqs.~(\ref{e57})--(\ref{e59})  can be solved to give\,\cite{haz}
\begin{align}\label{e67}
\delta\bar{n}_e(g,\hat{k})&=  \frac{-[1+g_2\,(G-g_1/2)]\,\bar{S}_0(g,\hat{k})+\bar{S}_2(g,\hat{k})/2}{(1+g)\,[(G-g_1/2)\,({\mit\Lambda}_D-g_1\,g_2)-g_1]},\\[0.5ex]
\bar{V}_e(g,\hat{k}) &=\frac{-(G-g_1/2)\,{\mit\Lambda}_D\,\xi\,\bar{S}_0(g,\hat{k})+g_1\,\xi\,\bar{S}_2(g,\hat{k})/2}{(1+g)\,[(G-g_1/2)\,({\mit\Lambda}_D-g_1\,g_2)-g_1]},\\[0.5ex]
\delta\bar{T}_e(g,\hat{k})&= \frac{{\mit\Lambda}_D\,\bar{S}_0(g,\hat{k})-({\mit\Lambda}_D-g_1\,g_2)\,\bar{S}_2(g,\hat{k})/2}{(1+g)\,[(G-g_1/2)\,({\mit\Lambda}_D-g_1\,g_2)-g_1]}.\label{e69}
\end{align}
The previous three equations specify the Fourier-Laplace transformed electron number density, drift velocity, and temperature
directly in terms of the particle and energy sources. 

\iffalse
\section{Results}
\subsection{Model Electron Particle and Energy Sources}
Suppose that the particle and energy sources have a Gaussian spatial dependence, with a maximum at $\hat{x}=0$, and a characteristic width $\sigma=\hat{\sigma}\,l_e$. Suppose,
further, that the sources  
are switched on suddenly at $\hat{t}=0$.  In other words, 
\begin{align}
S_0(\hat{t},\hat{x}) &= \frac{A_0}{\sqrt{2\pi}\,\hat{\sigma}}\,\exp\left(-\frac{\hat{x}^2}{2\,\hat{\sigma}^2}\right)H(\hat{t}),\\[0.5ex]
S_2(\hat{t},\hat{x}) &= \frac{A_2}{\sqrt{2\pi}\,\hat{\sigma}}\,\exp\left(-\frac{\hat{x}^2}{2\,\hat{\sigma}^2}\right)H(\hat{t}),
\end{align}
where $H(\hat{t})$ is a Heaviside step function, and $A_0$ and $A_2$ are arbitrary constants. 
Note that
\begin{align}
\int_{-\infty}^\infty S_0(\hat{t},\hat{x})\,d\hat{x}  & =A_0\,H(\hat{t}),\\[0.5ex]
\int_{-\infty}^\infty S_2(\hat{t},\hat{x})\,d\hat{x}  & =A_2\,H(\hat{t}).
\end{align}
It follows from Eq.~(\ref{e41x}) that 
\begin{align}
\bar{S}_0(g,\hat{k}) = \frac{A_0}{\sqrt{2\pi}\,g}\,\exp\left[-\frac{(\hat{k}\,\hat{\sigma})^2}{2}\right],\\[0.5ex]
\bar{S}_2(g,\hat{k}) = \frac{A_2}{\sqrt{2\pi}\,g}\,\exp\left[-\frac{(\hat{k}\,\hat{\sigma})^2}{2}\right].
\end{align}
Let
\begin{align}
K_{n,0}(g,\hat{k})&=   \frac{-1-g_2\,(G-g_1/2)}{g\,(1+g)\,[(G-g_1/2)\,({\mit\Lambda}_D-g_1\,g_2)-g_1]},\\[0.5ex]
K_{T,0}(g,\hat{k})&= \frac{{\mit\Lambda}_D}{g\,(1+g)\,[(G-g_1/2)\,({\mit\Lambda}_D-g_1\,g_2)-g_1]},\\[0.5ex]
K_{n,2}(g,\hat{k})&=   \frac{1}{2\,g\,(1+g)\,[(G-g_1/2)\,({\mit\Lambda}_D-g_1\,g_2)-g_1]},\\[0.5ex]
K_{T,2}(g,\hat{k})&= \frac{-{\mit\Lambda}_D+g_1\,g_2}{2\,g\,(1+g)\,[(G-g_1/2)\,({\mit\Lambda}_D-g_1\,g_2)-g_1]}.
\end{align}
and
\begin{align}
F_{n,0}(g,\hat{x})&=\frac{1}{\pi}\int_{0}^\infty K_{n,0}(g,\hat{k})\,\exp\left[-\frac{(\hat{k}\,\hat{\sigma})^2}{2}\right]\cos(\hat{k}\,\hat{x})\,d\hat{k},\\[0.5ex]
F_{T,0}(g,\hat{x})&=\frac{1}{\pi}\int_{0}^\infty K_{T,0}(g,\hat{k})\,\exp\left[-\frac{(\hat{k}\,\hat{\sigma})^2}{2}\right]\cos(\hat{k}\,\hat{x})\,d\hat{k},\\[0.5ex]
F_{n,2}(g,\hat{x})&=\frac{1}{\pi}\int_{0}^\infty K_{n,2}(g,\hat{k})\,\exp\left[-\frac{(\hat{k}\,\hat{\sigma})^2}{2}\right]\cos(\hat{k}\,\hat{x})\,d\hat{k},\\[0.5ex]
F_{T,2}(g,\hat{x})&=\frac{1}{\pi}\int_{0}^\infty K_{T,2}(g,\hat{k})\,\exp\left[-\frac{(\hat{k}\,\hat{\sigma})^2}{2}\right]\cos(\hat{k}\,\hat{x})\,d\hat{k}.
\end{align}
Here, we have made use of the easily proved results that if $\hat{k}\rightarrow -\hat{k}$ then $\xi\rightarrow -\xi$, $Z\rightarrow-Z$, $Z_0\rightarrow Z_0$,
$G\rightarrow G$, and ${\mit\Lambda}_D\rightarrow {\mit\Lambda}_D$.

If there is a unit amplitude particle source, but no energy source, so that $A_0=1$ and $A_2=0$, then Eqs.~(\ref{e41x}), (\ref{e67}), and (\ref{e69}) yield 
\begin{align}
\delta\hat{n}_e(\hat{t},\hat{x})&=\frac{{\rm e}^{\gamma\,\hat{t}}}{\pi}\int_{0}^{\infty}{\rm Re}[F_{n,0}(\gamma+{\rm i}\,y,\hat{x})]\,\cos(y\,\hat{t})\,dy\nonumber\\[0.5ex]
&\phantom{=}
 -\frac{{\rm e}^{\gamma\,\hat{t}}}{\pi}\int_{0}^{\infty}{\rm Im}[F_{n,0}(\gamma+{\rm i}\,y),\hat{x})]\,\sin(y\,\hat{t})\,dy,\\[0.5ex]
\delta\hat{T}_e(\hat{t},\hat{x})&=\frac{{\rm e}^{\gamma\,\hat{t}}}{\pi}\int_{0}^{\infty}{\rm Re}[F_{T,0}(\gamma+{\rm i}\,y,\hat{x})]\,\cos(y\,\hat{t})\,dy\nonumber\\[0.5ex]
&\phantom{=}
 -\frac{{\rm e}^{\gamma\,\hat{t}}}{\pi}\int_{0}^{\infty}{\rm Im}[F_{T,0}(\gamma+{\rm i}\,y),\hat{x})]\,\sin(y\,\hat{t})\,dy,
\end{align}
where $\gamma>0$.  
Note  $\delta n_e(\hat{t},-\hat{x})=\delta n_e(\hat{t},\hat{x})$, and $\delta\hat{T}_e(\hat{t},-\hat{x})=\delta \hat{T}_e(\hat{t},\hat{x})$. Here, use has been made of the easily proved results that if $g\rightarrow g^\ast$ then $K_{n,0}\rightarrow K_{n,0}^\ast$, 
and $K_{T,0}\rightarrow K_{T,0}^\ast$.
  Once we have determined $\delta\hat{n}_e(\hat{t},\hat{x})$ and $\delta\hat{T}_e(\hat{t},\hat{x})$ then the
normalized particle flux, $\hat{V}_e(\hat{t},\hat{x})$, and the normalized heat flux, $\hat{q}_e(\hat{t},\hat{x})$ can be obtained from the fluid
equations, (\ref{econt}) and (\ref{eenergy}). 

On the other hand, if there is a unit amplitude energy source, but no particle source, so that $A_0=0$ and $A_2=1$, then
\begin{align}
\delta\hat{n}_e(\hat{t},\hat{x})&=\frac{{\rm e}^{\gamma\,\hat{t}}}{\pi}\int_{0}^{\infty}{\rm Re}[F_{n,2}(\gamma+{\rm i}\,y,\hat{x})]\,\cos(y\,\hat{t})\,dy\nonumber\\[0.5ex]
&\phantom{=}
 -\frac{{\rm e}^{\gamma\,\hat{t}}}{\pi}\int_{0}^{\infty}{\rm Im}[F_{n,2}(\gamma+{\rm i}\,y),\hat{x})]\,\sin(y\,\hat{t})\,dy,\\[0.5ex]
\delta\hat{T}_e(\hat{t},\hat{x})&=\frac{{\rm e}^{\gamma\,\hat{t}}}{\pi}\int_{0}^{\infty}{\rm Re}[F_{T,2}(\gamma+{\rm i}\,y,\hat{x})]\,\cos(y\,\hat{t})\,dy\nonumber\\[0.5ex]
&\phantom{=}
 -\frac{{\rm e}^{\gamma\,\hat{t}}}{\pi}\int_{0}^{\infty}{\rm Im}[F_{T,2}(\gamma+{\rm i}\,y),\hat{x})]\,\sin(y\,\hat{t})\,dy.
\end{align}
Here, use has been made of the easily proved results that if $g\rightarrow g^\ast$ then $K_{n,0}\rightarrow K_{n,0}^\ast$, 
$K_{n,2}\rightarrow K_{n,2}^\ast$, 
and $K_{T,0}\rightarrow K_{T,2}^\ast$. As before, once we have determined $\delta\hat{n}_e(\hat{t},\hat{x})$ and $\delta\hat{T}_e(\hat{t},\hat{x})$ then the
normalized particle flux, $\hat{V}_e(\hat{t},\hat{x})$, and the normalized heat flux, $\hat{q}_e(\hat{t},\hat{x})$ can be determined from the fluid
equations, (\ref{econt}) and (\ref{eenergy}). 
\fi

\section*{Acknowledgements}
This research was directly funded by the U.S.\ Department of Energy, Office of Science, Office of Fusion Energy Sciences, under  contract DE-SC0021156. 

\section*{Data Availability Statement}
The digital data used in the figures in this paper can be obtained from the author upon reasonable request.

\section*{References}
\begin{thebibliography}{99}\baselineskip 5ex

\bibitem{haz} R.D.~Hazeltine, Phys.\ Plasmas {\bf 5}, 3282 (1998).

\bibitem{krook} P.L.~Bhatnagar, E.P.~Gross and M.~Krook, Phys.\ Rev.\ {\bf 94}, 511 (1954).

\bibitem{rf0} R.~Fitzpatrick, {\em Plasma Physics: An Introduction}, 2nd Edition. (CRC Press, Boca Raton FL, 2022.)

\bibitem{fc} B.D.~Fried and S.D.~Conte, {\em The Plasma Dispersion Function}. (Academic Press, New York NY, 1961.)

\bibitem{as} M.~Abramowitz and I.A.~Stegun, {\em Handbook of Mathematical Functions}. (Dover, New York NY, 1965).

\end{thebibliography}


\end{document}