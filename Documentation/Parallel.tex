\documentclass[12pt,prb,aps]{revtex4-1}
\usepackage {amsmath}
\usepackage{amssymb}
\pdfoutput = 1 
\usepackage {graphicx}
\newcommand{\bomega}{\mbox{\boldmath$\omega$}}
\allowdisplaybreaks

\begin{document}

\title{Time-Dependent Parallel Electron Energy Transport in  a Magnetized Plasma of Arbitrary Collisonality}
\author{T.~Wang and R.~Fitzpatrick\,\footnote{rfitzp@utexas.edu}}
\affiliation{Institute for Fusion Studies,  Department of Physics,  University of Texas at Austin,  Austin TX 78712, USA}
\maketitle

\section{Introduction}
In Hazeltine (1998),\cite{haz}  the steady-state transport of electron number density and energy parallel to the magnetic field
of a magnetized, weakly-coupled, electron-ion plasma of arbitrary collisionality is investigated  in slab geometry   by solving a simplified one-dimensional kinetic equation for the electron distribution function that employs a Bhatnagar-Gross-Krook (BGK)  electron-electron collision
operator.\cite{krook}  The resulting model is able to successfully reproduce standard results for the the electron heat flux in both the short mean-free-path and
the long mean-free-path limits. In the short mean-free-path limit, electron energy transport is found to be local and diffusive in nature, whereas the transport is found to be non-local and convective in the
long mean-free-path limit. 

The aim of this paper is to generalize the analysis of Hazeltine (1998) by  incorporating a model electron-ion collision operator,
 including a self-consistent calculation of the parallel electric field, and taking time dependence into account. The resulting enhanced model is used to investigate the transport of electron
 energy across a magnetic island chain in a tokamak plasma. 

\section{Fundamental Model}\label{s2}

\subsection{Electron Distribution Function}
Let $f_e(t,{\bf x},{\bf v})$ be the ensemble-averaged electron distribution function. Here, $t$ denotes time, ${\bf x}=(x_1,\, x_2,\, x_3)$
is a position vector,  $x_1$, $x_2$, $x_3$ are Cartesian coordinates that are defined such that the 
$x_3$-axis is parallel to the local equilibrium magnetic field, and ${\bf v}$ is the electron velocity. 
Let us write
\begin{equation}\label{e22}
f_e(t,{\bf x},{\bf v})= n_e\,F(v_1)\,F(v_2)\left[F(v_3)+ f (t,x_3,v_3)\right],
\end{equation}
where 
\begin{equation}
F(v) = \frac{\exp(-v^2/v_{t\,e}^{\,2})}{\pi^{1/2}\,v_{t\,e}},
\end{equation}
and 
$|f/F|\ll 1$. Here,  $n_e$ is the unperturbed electron number density, 
\begin{equation}
v_{t\,e} = \sqrt{\frac{2\,T_e}{m_e}}
\end{equation}
the electron thermal velocity, $m_e$  the electron mass,  and $T_e$ the unperturbed electron temperature (measured in energy units). Note that we are assuming that the electron distribution function remains relatively close to a Maxwellian distribution. 

\subsection{Electron-Electron Collision Operator}\label{see}
The electron-electron collision operator is modeled as a BGK operator:\,\cite{haz,krook}
\begin{align}
C_{ee}(f_e) &= -\nu_{ee}\,F(v_1)\,F(v_2)\,\biggr\{n_e\,F(v_3)+n_e\,f(t,x_3,v_3)\phantom{\frac{a}{b}}\nonumber\\[0.5ex]
&\phantom{=}\left.- \frac{[n_e+\delta n_e(t,x_3)]\,m_e^{\,1/2}}{\pi^{1/2}\,(2\,[T_e+\delta T_e(t,x_3)])^{1/2}}\,\exp\left[-\frac{[v_3-V_e(t,x_3)]^{2}\,m_e}{2\,[T_e+\delta T_e(t,x_3)]}\right]\right\}.
\end{align}
Here, $\nu_{ee}$ is the electron-electron collision frequency. Moreover, $|\delta n_e/n_e|\ll 1$, $|V_e/v_3|\ll 1$,  and $|\delta T_e/T_e|\ll 1$. It can be seen that the operator acts to relax the distribution function to the 
perturbed Maxwellian
\begin{equation}
F(v_1)\,F(v_2)\,\frac{[n_e+\delta n_e(t,x_3)]\,m_e^{\,1/2}}{\pi^{1/2}\,(2\,[T_e+\delta T_e(t,x_3)])^{1/2}}\,\exp\left[-\frac{[v_3-V_3(t,x_3)]^{2}\,m_e}{2\,[T_e+\delta T_e(t,x_3)]}\right].
\end{equation}
Note that we are working in an assumed common electron-ion rest frame.
Expanding the collision operator, and only retaining terms that are first order in perturbed quantities, we obtain
\begin{align}\label{e24}
C_{ee}(f_e) &=- \nu_{ee}\,n_e\,F(v_1)\,F(v_2)\,\left\{ f(t,x_3,v_3) - \left[\frac{\delta n_e(t,x_3)}{n_e} +\frac{V_e(t,x_3)}{v_{t\,e}} \,\frac{2\,v_3}{v_{t\,e}}\right.\right.\nonumber\\[0.5ex]
&\phantom{=}\left.\left.+\frac{\delta T_e(t,x_3)}{T_e}\left(\frac{v_3^{\,2}}{v_{t\,e}^{\,2}}-\frac{1}{2}\right)\right]F(v_3)\right\}.
\end{align}

Now, in order for the electron-electron collision operator to conserve the number of electrons, we require that
\begin{equation}
\int\!\!\int\!\!\int C_{ee}(f_e) \,d^3{\bf v} = 0,
\end{equation}
which yields
\begin{equation}\label{dne}
\frac{\delta n_e(t,x_3)}{n_e} =\int_{-\infty}^\infty f(t,x_3,v_3)\,dv_3.
\end{equation}
Here, $\delta n_e(t,x_3)$ is the perturbed electron number density. 

Likewise, in order for the electron-electron collision operator to conserve electron momentum, we need
\begin{equation}\label{e27}
\int\!\!\int\!\!\int {\bf v}\,C_{ee}(f_e)\, d^3{\bf v}= {\bf 0}.
\end{equation}
It is easily seen that 
$\int\!\!\int\!\!\int v_1\,C_{ee}(f_e)\,d^3{\bf v} =\int\!\!\int\!\!\int v_2\,C_{ee}(f_e)\,d^3{\bf v} = 0$.
Thus, we require 
\begin{equation}\label{eyy}
\int\!\!\int\!\!\int v_3\,C_{ee}(f_e)\, d^3{\bf v} = 0,
\end{equation}
which yields
\begin{equation}\label{e29}
 V_e(t,x_3) = \int_{-\infty}^\infty v_3\,f(t,x_3,v_3)\,dv_3.
\end{equation}
Here, $V_e(t,x_3)$ is the perturbed parallel drift velocity of the electrons with respect to the ions.

Finally, in order for the electron-electron collision operator to conserve electron energy, we require
\begin{equation}\label{e20}
\int\!\!\int\!\!\int v^2\,C_{ee}(f_e)\, d^3{\bf v} = 0.
\end{equation}
It is easily seen that 
$\int\!\!\int\!\!\int v_1^{\,2}\,C_{ee}(f_e) \,d^3{\bf v} = \int\!\!\int\!\!\int v_2^{\,2}\,C_{ee}(f_e)\,d^3{\bf v} =0$.
Thus, we need 
\begin{equation}\label{e36s}
\int\!\!\int\!\!\int v_3^{\,2}\,C_{ee}(f_e) \,d^3{\bf v} = 0,
\end{equation}
which yields
\begin{equation}\label{dte}
\frac{\delta T_e(t,x_3)}{T_e} = 2\,\int_{-\infty}^\infty \left(\frac{v_3^{\,2}}{v_{t\,e}^{\,2}}-\frac{1}{2}\right) f(t,x_3,v_3)\,dv_3.
\end{equation}
Here, $\delta T_e(t,x_3)$ is the perturbed electron temperature.
Thus, the electron-electron collision operator, (\ref{e24}), is now fully specified in terms of the perturbed electron distribution function,
$f(t,x_3,v_3)$. 

\subsection{Electron-Ion Collision Operator}
By analogy with the analysis in the previous subsection, our model electron-ion collision operator is written
\begin{align}
C_{ei}(f_e) &=- \nu_{ei}\,n_e\,F(v_1)\,F(v_2)\,\biggr\{ f(t,x_3,v_3) \nonumber\\[0.5ex]&\phantom{=}\left.- \left[\frac{\delta n_e(t,x_3)}{n_e} +\frac{\delta T_e(t,x_3)}{T_e}\,\left(\frac{v_3^{\,2}}{v_{t\,e}^{\,2}}-\frac{1}{2}\right)\right]F(v_3)\right\},
\end{align}
where $\nu_{ei}$ is the electron-ion collision frequency. Note that this collision operator conserves the number of electrons, as well as the electron energy (because the ions are treated as
infinitely massive with respect to the electrons), but does not conserve electron momentum (as a consequence of momentum transferred to the ions via collisions). Note, finally, that the
ion fluid is stationary in the infinite mass limit. 

\subsection{Electron Kinetic Equation}
The  ensemble-averaged electron kinetic equation that governs the transport of electron number density
and energy parallel to the magnetic field can be written\,\cite{haz,rf0}
\begin{equation}
\frac{\partial f_e}{\partial t}+v_3\,\frac{\partial f_e}{\partial x_3} -\frac{e}{m_e}\,E_3\,\frac{\partial f_e}{\partial v_3} = C_{ei}(f_e) + C_{ee}(f_e) + S({\bf x},{\bf v}).
\end{equation}
Here, we are assuming that the plasma is subject to a perturbed parallel electric
field, $E_3(t,x_3)$. Moreover, the source term in the kinetic equation takes the form 
\begin{equation}
S(t,{\bf x},{\bf v}) = n_e\,F(v_1)\,F(v_2)\,F(v_3)\left[S_0(t,x_3) +S_2(t,x_3)\left(\frac{v_3^{\,2}}{v_{t\,e}^{\,2}}-\frac{1}{2}\right) \right],
\end{equation}
where $S_0(t,x_3)$ represents a particle source, and $S_2(t,x_3)$ represents an energy source. 

Linearizing the kinetic equation, and integrating over $v_1$ and $v_2$, we obtain
\begin{equation}\label{e40}
\frac{\partial f}{\partial t}+v_3\,\frac{\partial f}{\partial x_3} - \langle C_{ei}( f)\rangle- \langle C_{ee}(f)\rangle = \left[
S_0 +S_1\,\frac{2\,v_3}{v_{t\,e}} +S_2 \left(\frac{v_3^{\,2}}{v_{t\,e}^{\,2}}-\frac{1}{2}\right)\right]F(v_3),
\end{equation}
where
\begin{align}\label{cee}
\langle C_{ee}(f)\rangle&= \frac{1}{n_e}\int_{-\infty}^\infty\!\int_{-\infty}^\infty C_{ee}(f_e)\,dv_1\,dv_2\nonumber\\[0.5ex]&= -\nu_{ee}\left\{f-\left[\frac{\delta n_e}{n_e}+\frac{ V_e}{v_{t\,e}}\,\frac{2\,v_3}{v_{t\,e}}
+\frac{\delta T_e}{T_e}\left(\frac{v_3^{\,2}}{v_{t\,e}^{\,2}}-\frac{1}{2}\right)\right]F(v_3)\right\},\\[0.5ex]
\langle C_{ei}(f)\rangle&= \frac{1}{n_e}\int_{-\infty}^\infty\!\int_{-\infty}^\infty C_{ei}(f_e)\,dv_1\,dv_2\nonumber\\[0.5ex]&=-\nu_{ei}\left\{f-\left[\frac{\delta n_e}{n_e}
+\frac{\delta T_e}{T_e}\left(\frac{v_3^{\,2}}{v_{t\,e}^{\,2}}-\frac{1}{2}\right)\right]F(v_3)\right\},\label{cei}\\[0.5ex]
S_1(t,x_3) &= - \frac{e\, E_3(t,x_3)}{m_e\,v_{t\,e}}.\label{s1}
\end{align}

\subsection{Poisson-Maxwell Equation}
Assuming that the ions constitute a uniform neutralizing background, the perturbed parallel electric field is related to the perturbed electron number
density according to
\begin{equation}\label{pois}
\frac{\partial  E_3}{\partial x_3}=-\frac{e\,\delta n_e(t,x_3)}{\epsilon_0}.
\end{equation}

\subsection{Heat Flux}
The flux of parallel electron kinetic energy is defined\,\cite{rf0}
\begin{equation}
{\bf q}_\parallel(t,{\bf x}) =  \int\!\int\!\int \frac{1}{2}\,m_e\,(v_3-V_e)^2\,({\bf v}-{\bf V})\,f_e(t,{\bf x},{\bf v})\,d^3{\bf v}.
\end{equation}
It is easily demonstrated that, to first order in small quantities, $q_{\parallel\,1}=q_{\parallel\,2}=0$, and
\begin{equation}\label{q3}
q_{\parallel\,3}(t,x_3) = \frac{1}{2}\,m_e\,n_e\int_{-\infty}^\infty v_3\left(v_3^{\,2}
-\frac{3}{2}\,v_{t\,e}^{\,2}\right)f(t,x_3,v_3)\,dv_3.
\end{equation}

\section{Fourier-Laplace Transform Solution of Electron Kinetic Equation}\label{s3}
\subsection{Normalization}
Let
\begin{align}
\nu_e= \nu_{ee}+\nu_{ei}
\end{align}
be the total electron collision frequency, and
let
\begin{equation}
l_e= \frac{v_{t\,e}}{\nu_e}
\end{equation}
be the electron mean-free-path between collisions. 
Let us adopt the following normalizations: 
$\hat{t} = \nu_{e}\,t$,
$\hat{x}= x_3/l_e$,
$u = v_3/v_{t\,e}$,
$\hat{f}= v_{t\,e}\,f$,
$\delta\hat{n}_e= \delta n_e/n_e$,
$\hat{V}_e= V_e/v_{t\,e}$,
$\delta\hat{T}_e= \delta T_e/T_e$,
$\hat{S}_0 = S_0/\nu_{e}$,
$\hat{S}_1 = S_1/\nu_{e}$,
$\hat{S}_2= S_2/\nu_{e}$, and
$\hat{q}_e=q_{\parallel\,3}/(n_e\,T_e\,v_{t\,e})$.
 
The electron kinetic equation, (\ref{e40}), takes the normalized form
\begin{align}\label{e37a}
\frac{\partial\hat{f}}{\partial \hat{t}}+u\,\frac{\partial\hat{f}}{\partial\hat{x}} + \hat{f}
= \left[(\delta\hat{n}_e+\hat{S}_0)+(\mu_e\,\hat{V}_e+\hat{S}_1)\,2\,u+(\delta\hat{T}_e+\hat{S}_2)\left(u^2-\frac{1}{2}\right)\right] F_M,
\end{align}
where
\begin{align}
F_M(u) &=\frac{\exp(-u^2)}{\pi^{1/2}},\\[0.5ex]
\mu_e&= \frac{\nu_{ee}}{\nu_{ee}+\nu_{ei}}.
\end{align}
Here, use has been made of Eqs.~(\ref{cee}) and (\ref{cei}). Furthermore, Eqs.~(\ref{dne}), (\ref{e29}), (\ref{dte}), and (\ref{q3}) yield
\begin{align}\label{e28}
\delta\hat{n}_e(\hat{t},\hat{x})&=\int_{-\infty}^\infty \hat{f}(\hat{t},\hat{x},u)\,du,\\[0.5ex]
\hat{V}_e(\hat{t},\hat{x})&= \int_{-\infty}^\infty u\,\hat{f}(\hat{t},\hat{x},u)\,du,\\[0.5ex]
\delta\hat{T}_e(\hat{t},\hat{x})&= 2\int_{-\infty}^\infty \left(u^2-\frac{1}{2}\right)f(\hat{t},\hat{x},u)\,du,\label{e30}\\[0.5ex]
\hat{q}_e(\hat{t},\hat{x})&= \int_{-\infty}^\infty u\left(u^2-\frac{3}{2}\right)\hat{f}(\hat{t},\hat{x},u)\,du.
\end{align}
Finally, Eqs.~(\ref{s1}) and (\ref{pois}) give
\begin{equation}\label{e44a}
2\,\hat{\lambda}_D^{\,2}\,\frac{\partial\hat{S}_1}{\partial \hat{x}} = \delta\hat{n}_e,
\end{equation}
where
\begin{equation}
\hat{\lambda}_D = \frac{\lambda_D}{l_e}
\end{equation}
and
\begin{equation}
\lambda_D = \left(\frac{\epsilon_0\,T_e}{n_e\,e^2}\right)^{1/2}
\end{equation}
is the Deybe length.\cite{rf0} Note that $\hat{\lambda}_D$ is necessarily a small parameter in a weakly-coupled plasma.\cite{rf0}

\subsection{Fluid Equations}
Taking $\int_{-\infty}^\infty (\ref{e37a})\,du$, we obtain the electron continuity equation,
\begin{equation}\label{econt}
\frac{\partial\delta\hat{n}_e}{\partial\hat{t}}+\frac{\partial\hat{V}_e}{\partial\hat{x}} = \hat{S}_0.
\end{equation}
Taking $\int_{-\infty}^\infty u\,(\ref{e37a})\,du$, we obtain the electron momentum conservation equation, 
\begin{equation}\label{eforce}
\frac{\partial\hat{V}_e}{\partial\hat{t}}+\frac{1}{2}\,\frac{\partial}{\partial\hat{x}}(\delta\hat{n}_e+\delta\hat{T}_e) + (1-\mu_e)\,\hat{V}_e = \hat{S}_1.
\end{equation}
Finally, taking $2\,\int_{-\infty}^\infty (u^2-1/2)\,(\ref{e37a})\,du$, we obtain
 the electron  energy conservation equation,
\begin{equation}\label{eenergy}
\frac{\partial\delta T_e}{\partial\hat{t}} +2\,\frac{\partial}{\partial\hat{x}}(\hat{V}_e+\hat{q}_e)= \hat{S}_2.
\end{equation}


\subsection{Fourier-Laplace Transformation}
Let
\begin{align}
\bar{f}(g,\hat{k},u) &= \frac{1}{\sqrt{2\pi}}\int_{-\infty}^\infty\left(\int_0^\infty \hat{f}(\hat{t},\hat{x},u)\,{\rm e}^{-g\,\hat{t}}\,d\hat{t}\right){\rm e}^{-{\rm i}\,\hat{k}\,\hat{x}}\,d\hat{x},\\[0.5ex]
\delta\bar{n}_e(g,\hat{k}) &=\frac{1}{\sqrt{2\pi}} \int_{-\infty}^\infty\left(\int_0^\infty \delta \hat{n}_e(\hat{t},\hat{x})\,{\rm e}^{-g\,\hat{t}}\,d\hat{t}\right){\rm e}^{-{\rm i}\,\hat{k}\,\hat{x}}\,d\hat{x},
\label{e41x}
\end{align}
et cetera.
Here,
$\hat{k}= k\,l_e$,
where $k$ is the unormalized wavenumber.
If we operate on Eqs.~(\ref{e37a}) and (\ref{e44a}) with $\int_{-\infty}^\infty\int_0^\infty[(\cdots)\,{\rm e}^{-g\,\hat{t}}\,d\hat{t}]\,{\rm e}^{-{\rm i}\,\hat{k}\,\hat{x}}\,d\hat{x}$, 
and combine the
resulting equations, then we obtain
\begin{align}\label{e55x}
(g+{\rm i}\,\hat{k}\,u+1)\,\bar{f}&= \left[(\delta\bar{n}_e+\bar{S}_0)+\left(\mu_e\,\bar{V}_e+\frac{\delta\bar{n}_e}{2\,{\rm i}\,\hat{k}\,\hat{\lambda}_D^{\,2}}\right)2\,u+(\delta\bar{T}_e+\bar{S}_2)\left(u^2-\frac{1}{2}\right)\right] F_M.
\end{align}
Here, we are assuming that all perturbed quantities are zero for $\hat{t}<0$. 

\subsection{Fourier-Laplace Transformed Fluid Equations}
Taking $\int_{-\infty}^\infty (\ref{e55x})\,du$, we obtain the Fourier-Laplace transformed electron continuity equation,
\begin{equation}\label{e34}
g\,\delta\bar{n}_e+{\rm i}\,\hat{k}\,\bar{V}_e =\bar{S}_0.
\end{equation}
Taking $\int_{-\infty}^\infty u\,(\ref{e55x})\,du$, we obtain the Fourier-Laplace transformed electron momentum conservation equation,
\begin{equation}\label{e35}
(g+1-\mu_e)\,\bar{V}_e +\frac{{\rm i}\,\hat{k}}{2}\left({\mit\Lambda}_D\,\delta\bar{n}_e+\delta\bar{T}_e\right)=0,
\end{equation}
where
\begin{equation}
{\mit\Lambda}_D (\hat{k})= \frac{1+(\hat{k}\,\hat{\lambda}_D)^2}{(\hat{k}\,\hat{\lambda}_D)^2}= \frac{1+(k\,\lambda_D)^2}{(k\,\lambda_D)^2}.
\end{equation}
Finally, taking $2\int_{-\infty}^\infty(u^2-1/2)\,(\ref{e55x})\,du$, we obtain the Fourier-Laplace transformed electron energy conservation equation,
\begin{equation}\label{e36}
g\,\delta\bar{T}_e+ 2\,{\rm i}\,\hat{k}\,(\bar{V}_e+\bar{q}_e) =\bar{S}_2.
\end{equation}

\subsection{Reformulation}
Equation~(\ref{e55x}) can be rearranged to give
\begin{align}\label{e37}
\bar{f}(g,\hat{k},u)&=(1+g)^{-1}\left\{\left(\delta\bar{n}_e+\bar{S}_0\right)+\left[\mu_e\,\bar{V}_e+\frac{(1+g)\,({\mit\Lambda}_D-1)\,\delta\bar{n}_e}{2\,\xi}\right]2\,u
\right.\nonumber\\[0.5ex]
&\phantom{=}\left.+\left(\delta\bar{T}_e+\bar{S}_2\right)\left(u^2-\frac{1}{2}\right)\right\}\left(\frac{-\xi}{u-\xi}\right)F_M,
\end{align}
where
\begin{equation}\label{exi}
\xi (g,\hat{k})= \frac{{\rm i}\,(1+g)}{\hat{k}}= \frac{{\rm i}\,(1+g)}{k\,l_e}.
\end{equation}
Likewise, the fluid equations, (\ref{e34}), (\ref{e35}), and (\ref{e36}), can be re-expressed in the forms
\begin{align}\label{e39}
\delta\bar{n}_e+\bar{S}_0&= (1+g)\left(\delta\bar{n}_e-\xi^{-1}\,\bar{V}_e\right),\\[0.5ex]
\mu_e\,\bar{V}_e+\frac{(1+g)\,({\mit\Lambda}_D-1)\,\delta\bar{n}_e}{2\,\xi}&= (1+g)\left[\bar{V}_e-\frac{\xi^{-1}}{2}\,(\delta\bar{n}_e+\delta\bar{T}_e)\right],\\[0.5ex]
\delta\bar{T}_e+\bar{S}_2&= (1+g)\left[\delta \bar{T}_e-2\,\xi^{-1}\,(\bar{V}_e+\bar{q}_e)\right].\label{e41}
\end{align}
It follows that
\begin{align}\label{e42}
\xi\left[\mu_e\,\bar{V}_e+\frac{(1+g)\,({\mit\Lambda}_D-1)\,\delta\bar{n}_e}{2\,\xi}\right]& = g\,\xi^2\,\delta\bar{n}_e-\frac{1}{2}\,(1+g)\,(\delta \bar{n}_e+\delta\bar{T}_e)
-\xi^2\,\hat{S}_0,\\[0.5ex]
\bar{q}_e&= (1+g)^{-1}\,\xi\left[-g\left(\delta\bar{n}_e-\frac{\delta\bar{T}_e}{2}\right)+\left(\bar{S}_0-\frac{\bar{S}_2}{2}\right)\right].\label{e43}
\end{align}

\subsection{Modified Plasma Dispersion Function}
Let
\begin{equation}\label{e44}
Z_n(\xi)= \int_{-\infty}^\infty u^n\left(\frac{-\xi}{u-\xi}\right)F_M(u)\,du.
\end{equation}
It is easily demonstrated that 
\begin{equation}
Z_{n+1}= \xi\,(Z_n-I_{n}),
\end{equation}
where
\begin{equation}
I_n = \frac{1}{\pi^{1/2}}\int_{-\infty}^\infty u^n\,\exp(-u^2)\,du.
\end{equation}
Now, $I_0=1$, and $I_1=0$, so
\begin{align}\label{ez1}
Z_1 &= \xi\,(Z_0-I_0) = \xi\,Z_0-\xi,\\[0.5ex]
Z_2&= \xi\,(Z_1-I_1) =  \xi^2\,Z_0-\xi^2.\label{ez2}
\end{align}
Note that 
\begin{equation}\label{z0}
Z_0(\xi)= -\xi\,\bar{Z}(\xi),
\end{equation}
where
\begin{equation}\label{z1}
\bar{Z}(\xi) = \frac{1}{\pi^{1/2}}\int_{-\infty}^\infty \frac{{\rm e}^{-u^2}}{u-\xi}\,du
\end{equation}
is related to the plasma dispersion function.\cite{rf0,fc}
In fact, it can be shown that\,\cite{rf0}
\begin{equation}
\bar{Z}(\xi)= {\rm i}\,\pi^{1/2}\,w(\xi)
\end{equation}
for ${\rm Im}(\xi)>0$, and 
\begin{equation}
\bar{Z}(\xi)={\rm i}\,\pi^{1/2}\,w(\xi)-2\,{\rm i}\,\pi^{1/2}\,\exp(-\xi^2) 
\end{equation}
for ${\rm Im}(\xi)<0$.
Here,
\begin{equation}
w(\xi)= \exp(-\xi^2)\,{\rm erfc}(-{\rm i}\,\xi)
\end{equation}
is a so-called Faddeeva function (alternatively known as a Kramp function),\cite{as} and 
${\rm erfc}(z)$ is the complementary error function.\cite{as} 

Now,\cite{rf0,as}
\begin{equation}
w(\xi)=1+\frac{2\,{\rm i}\,\xi}{\pi^{1/2}}+{\cal O}(\xi^2)
\end{equation}
in the limit $|\xi|\ll 1$, whereas
\begin{equation}
w(\xi) = \sigma\,\exp(-\xi^2)+ \frac{{\rm i}}{\pi^{1/2}}\left[\frac{1}{\xi} + \frac{1}{2\,\xi^3}+\frac{3}{4\,\xi^5}+\frac{15}{8\,\xi^7}+
\frac{105}{16\,\xi^{9}}+{\cal O}\left(\frac{1}{\xi^{11}}\right)\right]
\end{equation}
in the limit $|\xi|\rightarrow\infty$. 
Here,
\begin{equation}
\sigma = \left\{\begin{array}{lll} 0 &~~~~&\mbox{$\xi_i>|\xi_r|^{-1}$}\\[0.25ex]
1 &&\mbox{$|\xi_i| < |\xi_r|^{-1}$}\\[0.25ex]
2&& \mbox{$\xi_i< -|\xi_r|^{-1}$}
\end{array}
\right.,
\end{equation}
where $\xi=\xi_r+{\rm i}\,\xi_i$, and $\xi_r$ and $\xi_i$ are both real. 
It follows that
\begin{equation}\label{e63x}
Z_0(\xi) = -{\rm i}\,\pi^{1/2}\,{\rm sgn}(\xi_i)\,\xi +2\,\xi^2+{\cal O}(\xi^3)
\end{equation}
in the limit $|\xi|\ll 1$, whereas 
\begin{equation}\label{e64x}
Z_0(\xi) =- {\rm i}\,\pi^{1/2}\,\sigma'\,\xi\,\exp(-\xi^2) + 1 + \frac{1}{2\,\xi^2}+\frac{3}{4\,\xi^4}+\frac{15}{8\,\xi^6}+\frac{105}{16\,\xi^8}+{\cal O}\left(\frac{1}{\xi^{10}}\right)
\end{equation}
in the limit $|\xi|\gg 1$,
where
\begin{equation}
\sigma' = \left\{\begin{array}{lll} 0 &~~~~&\mbox{$|\xi_i|>|\xi_r|^{-1}$}\\[0.25ex]
1 &&\mbox{$0<\xi_i < |\xi_r|^{-1}$}\\[0.25ex]
-1&& \mbox{$-|\xi_r|^{-1}<\xi_i<0$}
\end{array}
\right..
\end{equation}

\subsection{Fourier-Laplace Transformed Electron Heat Flux}
Equations~(\ref{e28}), (\ref{e37}), and (\ref{e44}) can be combined to give
\begin{align}\label{e51}
(1+g)\,\delta\bar{n}_e&=\left[(\delta\bar{n}_e+\bar{S}_0)-\frac{1}{2}\,(\delta\bar{T}_e+\bar{S}_2)\right]Z_0+\left[\mu_e\,\bar{V}_e+\frac{(1+g)\,({\mit\Lambda}_D-1)\,\delta\bar{n}_e}{2\,\xi}\right]2\,Z_1\nonumber\\[0.5ex]
&\phantom{=}
+(\delta\bar{T}_e+\bar{S}_2)\,Z_2.
\end{align}
Equations~(\ref{e42}), (\ref{ez1}), (\ref{ez2}), and (\ref{e51})
yield
\begin{equation}\label{e54}
\delta\bar{T}_e = \frac{[2\,\xi^2-(2\,\xi^2-1)\,Z_0]\,[\bar{S}_0-\bar{S}_2/2-g\,\delta\bar{n}_e]}{
(\xi^2-1-g)-(\xi^2-3/2-g)\,Z_0}.
\end{equation}
Finally, Eqs.~(\ref{e43}) and (\ref{e54}) give\,\cite{haz}
\begin{equation}\label{e55}
\bar{q}_e(g,\hat{k}) = \xi\,G(\xi)\,\delta\bar{T}_e(g,\hat{k}),
\end{equation}
where
\begin{equation}\label{e56}
G(\xi) = \frac{(\xi^2-1)-(\xi^2-3/2)\,Z_0}{2\,\xi^2-(2\,\xi^2-1)\,Z_0}.
\end{equation}
Note that the electron heat flux only depends on the perturbed electron temperature, and is independent of both the
perturbed electron number density and the electron drift velocity. 
It follows from Eqs.~(\ref{e63x}) and (\ref{e64x}) that
\begin{equation}
G(\xi) =-{\rm i}\,\frac{{\rm sgn}(\xi_i)}{\pi^{1/2}\,\xi}+{\cal O}(1)
\end{equation}
in the limit $|\xi|\ll 1$, and 
\begin{equation}
G(\xi)= \frac{3}{4\,\xi^2} + \frac{3}{2\,\xi^4} + {\cal O}(\xi^{-6}) 
\end{equation}
in the limit $|\xi|\gg 1$. 

\subsection{Fourier-Laplace Transformed Electron Fluid Quantities}
The Fourier-Laplace transformed fluid equations, (\ref{e39})--(\ref{e41}), combined with Eq.~(\ref{e55}), yield
\begin{align}\label{e57}
g_1\,\delta\bar{n}_e-\xi^{-1}\,\bar{V}_e&= \frac{\bar{S}_0}{1+g},\\[0.5ex]
-{\mit\Lambda}_D\,\delta\bar{n}_e+g_2\,\xi^{-1}\,\bar{V}_e-\delta\bar{T}_e&=0,\\[0.5ex]
-\xi^{-1}\,\bar{V}_e-\left(G-\frac{g_1}{2}\right)\delta\bar{T}_e&= \frac{\bar{S}_2}{2\,(1+g)},\label{e59}
\end{align}
where
\begin{align}
g_1(g)&= \frac{g}{1+g},\\[0.5ex]
g_2(g,\hat{k})&=\frac{2\,(1-\mu_e+g)\,\xi^2}{1+g}.
\end{align}
Finally, Eqs.~(\ref{e57})--(\ref{e59})  can be solved to give\,\cite{haz}
\begin{align}\label{e67}
\delta\bar{n}_e(g,\hat{k})&=  \frac{-[1+g_2\,(G-g_1/2)]\,\bar{S}_0(g,\hat{k})+\bar{S}_2(g,\hat{k})/2}{(1+g)\,[(G-g_1/2)\,({\mit\Lambda}_D-g_1\,g_2)-g_1]},\\[0.5ex]
\bar{V}_e(g,\hat{k}) &=\frac{-(G-g_1/2)\,{\mit\Lambda}_D\,\xi\,\bar{S}_0(g,\hat{k})+g_1\,\xi\,\bar{S}_2(g,\hat{k})/2}{(1+g)\,[(G-g_1/2)\,({\mit\Lambda}_D-g_1\,g_2)-g_1]},\\[0.5ex]
\delta\bar{T}_e(g,\hat{k})&= \frac{{\mit\Lambda}_D\,\bar{S}_0(g,\hat{k})-({\mit\Lambda}_D-g_1\,g_2)\,\bar{S}_2(g,\hat{k})/2}{(1+g)\,[(G-g_1/2)\,({\mit\Lambda}_D-g_1\,g_2)-g_1]}.\label{e69}
\end{align}
The previous three equations specify the Fourier-Laplace transformed electron number density, drift velocity, and temperature
directly in terms of the particle and energy sources. 

\section{Electron Energy Transport across a Magnetic Island Chain}

\subsection{Introduction}
Consider a tearing mode  in a tokamak plasma.\cite{fkr} Suppose that the mode possesses $m$ periods in the poloidal direction, and $n$ periods in the toroidal direction. As is well-known, the
mode resonates with the equilibrium magnetic field at the so-called ``rational surface'' whose minor radius, $r_s$, satisfies $q(r_s)=m/n$, where $q(r)$ is the
safety-factor profile, and $r$ denotes the minor radius of equilibrium magnetic flux-surfaces.\cite{wesson} The tearing mode reconnects magnetic flux at the rational surface, in the process opening up a magnetic island chain, centered on the surface,  which also possesses $m$ periods in the poloidal direction, and $n$ periods in the toroidal direction.\cite{ruth}  Let $W$ be the full radial width of the island chain. 

As is well-known, if $W$ exceeds a fairly small critical width, $W_c$, then the equilibrium electron temperature
gradient is flattened in the region lying within the island chain's magnetic separatrix. In this case, the perpendicular electron energy flux across the island chain, driven by radial gradients in the equilibrium electron temperature, is diverted   along magnetic field-lines
in a thin boundary layer that lies on the island separatrix. See Fig.~\ref{fig1}. Let $\delta$ be the radial width of the boundary layer.

 The critical island width, $W_c$, is defined as the island width at which $\delta = W$. If
$W>W_c$ then the electron temperature profile is flattened by the island chain, and the energy transport across the chain is predominately parallel to magnetic field-lines. On the other hand,  if $W<W_c$ then the electron temperature profile is unaffected by the island chain, and the energy transport across the chain is predominately perpendicular to magnetic field-lines.\cite{rf} 

We wish to employ the theory developed Sects.~\ref{s2} and \ref{s3} to calculate the critical
island width at arbitrary collisionality. (In reality, parallel electron energy transport in tokamak plasmas does not lie in the short mean-free-path regime.) We also wish to calculate the critical island width in the case
in which the energy flow across the island chain oscillates at some frequency $\omega$. 

Our calculation takes place in the simplified island geometry pictured in Fig.~\ref{fig2}. Here, the separatrix boundary layer has been split into two straight boundary layers which represent the inner
(in $r$) and outer sides of the magnetic separatrix.  Electron energy flows parallel to magnetic field-lines within these boundary layers. Furthermore, any energy that flows out of the ends of the inner boundary layer is
fed into the corresponding end of the outer boundary layer, and vice versa. 

\subsection{Connection Length}
Let $x_3$ measure distance along magnetic field-lines in the separatrix boundary layers. The equation of the field-lines is\,\cite{rf}
\begin{equation}
\frac{dx_3}{d\zeta} =\frac{R_0\,r_s}{n\,s_s}\,\frac{1}{|r-r_s|},
\end{equation}
where $s_s = (d\ln q/\ln s)_{r_s}$ is the magnetic shear at the rational surface, and $R_0$ is the plasma minor radius. Here, $\zeta=m\,\theta-n\,\phi$ is a helical angle. 
 Now, on the magnetic separatrix,\cite{rf}
\begin{equation}
|r-r_s| = \frac{W}{2}\,\sin\left(\frac{\zeta}{2}\right).
\end{equation}
Here, the island O-point lies at $\zeta=\pi$, whereas the X-points lie at $\zeta=0$, $2\pi$. The average value of $|r-r_s|$ in the boundary layers is
\begin{equation}
\langle |r-r_s|\rangle  =\frac{W}{2}\,\int_0^\pi \sin\left(\frac{\zeta}{2}\right)\,\frac{d\zeta}{\pi}= \frac{W}{\pi}.
\end{equation}
Hence, the so-called ``connection length", which is defined as the length of the boundary layers parallel to magnetic field-lines, is
\begin{equation}
L = \int_0^{2\pi}\frac{dx_3}{d\zeta}\,d\zeta = \frac{R_0\,r_s}{n\,s_s}\,\frac{2\pi}{\langle|r-r_s|\rangle} = \frac{R_0\,r_s}{n\,s_s}\,\frac{2\pi^2}{W}.
\end{equation}
Here, the island O-point lies at $x_3=0$, whereas the X-points lie at $x_3=\pm L/2$. Note that the two ends of the boundary layers lie at the X-points. See Fig.~\ref{fig2}.

\subsection{Temperature Perturbation}
Consider the inner separatrix boundary layer. Suppose that the perturbed electron temperature on the inner (in $r$) side of the layer is
\begin{equation}
\delta T_{e\,{\rm in}}(t,x_3) =\delta T_{e\,0}\,\cos\left(\pi\,\frac{x_3}{L}\right)\cos(\omega \,t).
\end{equation}
Suppose that the layer is much thinner (in $r$) than the island, which implies that the perturbed electron temperature is zero inside the island. Thus, the perturbed electron
temperature on the outer side of the layer is 
\begin{equation}
\delta T_{e\,{\rm out}}(t,x) =0.
\end{equation}
Finally, let 
\begin{equation}
\delta T_{e\,}(t,x_3) =\delta T_{e\,0}\,\cos\left(\pi\,\frac{x_3}{L}\right)\cos(\omega \,t-\alpha)
\end{equation}
be the average perturbed electron temperature within the layer. Here, $\alpha$ represents a phase-lag between the average temperature and the driving temperature on the
inner side of the layer. Moreover, $\omega>0$ is the oscillation frequency of the temperature perturbation. Finally, $\delta T_{e\,0}$ represents the small temperature difference 
between the middle  and the two ends of the layer that is responsible for driving the flow of energy around the island. Note that the perturbed temperature in the outer boundary layer is
minus that in the inner boundary layer. 

\subsection{Parallel Electron Heat Flux}
The normalized Fourier-Laplace transformed perturbed electron temperature in the inner boundary layer is
\begin{align}
\delta\bar{T}_e(g,\hat{k}) &= \frac{1}{\sqrt{2\pi}}\int_{-\infty}^\infty \left(\int_0^\infty \delta \hat{T}_e(\hat{x},\hat{t})\,{\rm e}^{-g\,\hat{t}}\,d\hat{t}\right){\rm e}^{-{\rm i}\,\hat{k}\,\hat{x}}\,d\hat{x}
\nonumber\\[0.5ex]
&=\sqrt{\frac{\pi}{2}}\, \delta\hat{T}_{e\,0}\left[\delta (\hat{k}-\hat{k_0})+ \delta(\hat{k}+\hat{k}_0)\right]\frac{1}{2}\left(
\frac{{\rm e}^{-{\rm i}\,\alpha}}{g-{\rm i}\,\hat{\omega}}+ \frac{{\rm e}^{\,{\rm i}\,\alpha}}{g+\hat{\rm i}\,\hat{\omega}}\right).
\end{align}
Here, $\hat{x}=x_3/l_e$, $\hat{L}=L/l_e$, $\hat{t}=\nu_e\,t$, $\hat{\omega}=\omega/\nu_e$, $\delta \hat{T} = \delta T_e/T_e$, $\delta \hat{T}_{e\,0} = \delta T_{e\,0}/T_e$, and
$\hat{k}_0=\pi/\hat{L}$. Moreover, $\nu_e$, $l_e$, and $T_e$ are the total electron collision frequency, the electron mean-free-path between collisions, and the equilibrium electron temperarture,
respectively, at the rational surface. 

Now, according to Eq.~(\ref{e55}), the Fourier-Laplace transformed normalized parallel electron heat flux in the inner boundary layer is
\begin{equation}
\bar{q}_e(g,\hat{k}) = \xi\,G(\xi)\,\delta\bar{T}_e(g,\hat{k}),
\end{equation}
where [see Eq.~(\ref{exi})]
\begin{equation}
\xi= \frac{{\rm i}\,(1+g)}{\hat{k}},
\end{equation}
and the function $G(\xi)$ is defined in Eqs.~(\ref{z0}), (\ref{z1}), and (\ref{e56}). 

Let
\begin{equation}
G(\xi) = G_r(\xi_r,\xi_i) + {\rm i}\,(\xi_r,\xi_i),
\end{equation}
where
\begin{align}
\xi_r &= \frac{\hat{\omega}}{\hat{k}_0},\\[0.5ex]
\xi_i&= \frac{1}{\hat{k}_0},
\end{align}
and $G_r$ and $G_i$ are real. It is easily demonstrated that
\begin{align}
G_r(\xi_r,\xi_i)&= G_r(-\xi_r,\xi_i)= G_r(\xi_r,-\xi_i)=G_r(-\xi_r,-\xi_i),\\[0.5ex]
G_i(\xi_r,\xi_i)&=- G_i(-\xi_r,\xi_i)= -G_i(\xi_r,-\xi_i)=G_i(-\xi_r,-\xi_i).
\end{align}
The time-asymptotic normalized parallel heat flux in real space,
\begin{equation}
\hat{q}_e(\hat{t},\hat{x}) = \lim_{\hat{t}\rightarrow\infty}\frac{1}{\sqrt{2\pi}}\int_{-\infty}^{\infty}
\left(\frac{1}{2\pi\,{\rm i}}\int_C \bar{q}_e(g,\hat{k})\,{\rm e}^{\,g\,\hat{t}}\,dg\right){\rm e}^{\,{\rm i}\,\hat{k}\,\hat{x}}\,d\hat{k},
\end{equation}
where $C$ is the Bromwich contour, can be shown to take the form 
\begin{align}
\hat{q}_e(\hat{t},\hat{x}) &= \delta\hat{T}_{e\,0}\,\{\left[-\cos\alpha\,(-\xi_r\,G_r+\xi_i\,G_i)+\sin\alpha\,(\xi_i\,G_r+\xi_r\,G_i)\right]\sin(\hat{\omega}\,t)\nonumber\\[0.5ex]
&\phantom{=}+\left[\sin\alpha\,(-\xi_r\,G_r+\xi_i\,G_i)-\cos\alpha\,(\xi_i\,G_r+\xi_r\,G_i)\right]\cos(\hat{\omega}\,t)\}\sin(\hat{k}_0\,\hat{x}).
\end{align}
Here, $\hat{q}_e = q_e/(n_e\,T_e\,v_{t\,e})$, where $n_e$ is the equilibrium electron number density at the rational surface, and $v_{t\,e}= (2\,T_e/m_e)^{1/2}= l_e\,\nu_e$. 
Moreover, $G_r= {\rm Re}[G(\xi_r,\xi_i)]$ and $G_i= {\rm Im}[G(\xi_r,\xi_i)]$.

\subsection{Energy Conservation}
The net parallel electron heat flux in the inner separatrix boundary layer is
\begin{equation}
Q_\parallel (\hat{t},\hat{x})= n_e\,T_e\,v_{t\,e}\,\hat{q}_e(\hat{x},\hat{t})\,\delta.
\end{equation}
The perpendicular heat flux into the section of the layer that lies between $x=0$ and $x=x$ is
\begin{align}
Q_\perp (\hat{t},\hat{x}) &= \frac{\kappa_\perp}{\delta} \int_0^x\delta T_{e\,{\rm in}}(\hat{x},\hat{t})\,dx\nonumber\\[0.5ex]
&= \frac{\kappa_\perp\,\delta T_{e\,0}}{\hat{\delta}\,\hat{k}_0}\,\cos(\hat{\omega}\,\hat{t})\,\sin(\hat{k}_0\,\hat{x}),
\end{align}
where $\kappa_\perp$ is the perpendicular electron thermal conductivity at the rational surface, and $\hat{\delta}=\delta/l_e$. Finally, the  electron thermal energy in the section of the layer is
\begin{align}
{\cal E}(\hat{t},\hat{x})&=\frac{1}{2}\int_0^x n_e\,\delta T_e(\hat{x},\hat{t})\,dx\,\delta\nonumber\\[0.5ex]
& = \frac{n_e\,\delta\,l_e\,\delta T_{e\,0}}{2\,\hat{k}}\left[\cos\alpha\,\cos(\hat{\omega}\,\hat{t})-\sin\alpha\,\sin(\hat{\omega}\,\hat{t})\right]\,
\sin(\hat{k}_0\,\hat{x}).
\end{align}

Energy conservation in the layer requires that [see Eq.~(\ref{eenergy})]
\begin{equation}
\frac{\partial{\cal E}}{\partial t} = Q_\perp-Q_\parallel,
\end{equation}
which implies that
\begin{align}
\cos\alpha\left(\frac{1}{2}\,\xi_r-\xi_r\,G_r+\xi_i\,G_i\right)+\sin\alpha\,(\xi_i\,G_r+\xi_r\,G_i) &= 0,\\[0.5ex]
\sin\alpha\left(\frac{1}{2}\,\xi_r-\xi_r\,G_r+\xi_i\,G_i\right)-\cos\alpha\,(\xi_i\,G_r+\xi_r\,G_i)&= \frac{\hat{\kappa}_\perp}{\hat{\delta}^{\,2}\,\hat{k}_0},
\end{align}
where
\begin{equation}
\hat{\kappa}_\perp =\frac{\kappa_\perp}{n_e\,v_{t\,e}\,l_e}.
\end{equation}
Thus, it follows that
\begin{equation}
\tan\alpha = \frac{\hat{\omega}/2 -\hat{\omega}\,G_r + G_i}{-G_r-\hat{\omega}\,G_i},
\end{equation}
and
\begin{equation}
\hat{\delta}^{\,2} = \frac{\hat{\kappa}_\perp}{[(\hat{\omega}/2 - \hat{\omega}\,G_r+G_i)^2 + (-G_r-\hat{\omega}\,G_i)^2]^{1/2}}.
\end{equation}

\section*{Acknowledgements}
This research was directly funded by the U.S.\ Department of Energy, Office of Science, Office of Fusion Energy Sciences, under  contract DE-SC0021156. 

\section*{Data Availability Statement}
The digital data used in the figures in this paper can be obtained from the author upon reasonable request.

\section*{References}
\begin{thebibliography}{99}\baselineskip 5ex

\bibitem{haz} R.D.~Hazeltine, Phys.\ Plasmas {\bf 5}, 3282 (1998).

\bibitem{krook} P.L.~Bhatnagar, E.P.~Gross and M.~Krook, Phys.\ Rev.\ {\bf 94}, 511 (1954).

\bibitem{rf0} R.~Fitzpatrick, {\em Plasma Physics: An Introduction}, 2nd Ed. (CRC Press, Boca Raton FL, 2022.)

\bibitem{fc} B.D.~Fried and S.D.~Conte, {\em The Plasma Dispersion Function}. (Academic Press, New York NY, 1961.)

\bibitem{as} M.~Abramowitz and I.A.~Stegun, {\em Handbook of Mathematical Functions}. (Dover, New York NY, 1965.)

\bibitem{fkr} H.P.~Furth,  J.~Killeen and M.N.~Rosenbluth,  Phys.\ Fluids {\bf 6}, 459 (1963).

\bibitem{wesson} J.A.~Wesson, {\em Tokamaks}, 4th Ed. (Oxford University Press, Oxford UK, 2011.)

\bibitem{ruth} P.H.~Rutherford, Phys.\ Fluids {\bf 16}, 1906 (1973).

\bibitem{rf} R. Fitzpatrick, Phys.\ Plasmas {\bf 2}, 825 (1995).

\end{thebibliography}

\begin{figure}
\centerline{\includegraphics[width=1.0\textwidth]{Island.eps}}
\caption{True geometry of electron energy transport across a magnetic island. Here, $\zeta= m\,\theta-n\,\phi$, where $\theta$ is a poloidal angle, and $\phi$ a
toroidal angle.  \label{fig1}}
\end{figure}

\begin{figure}
\centerline{\includegraphics[width=1.0\textwidth]{Island1.eps}}
\caption{Simplified geometry of electron energy transport across a magnetic island. Here, $x$ measures distance along magnetic field-lines.  \label{fig2}}
\end{figure}




\end{document}